\documentclass[a4paper, 11pt]{article}
\usepackage{comment} % enables the use of multi-line comments (\ifx \fi) 
\usepackage{fullpage} % changes the margin
\usepackage{graphicx}
\usepackage{subfigure}
\usepackage{amsmath}
\usepackage{amsfonts}



\begin{document}
\noindent
\large\textbf{Homework 1} \hfill \textbf{Yen-Lin Chen} \\
\normalsize ECE 5412 \hfill yc2253@cornell.edu \\
Fall 2019 \hfill Due: 09/31/19\\





\section*{Problem 17}

The CDF is given as 

\begin{equation}
F(x)= 
\begin{cases}
    \frac{1-e^{-2x}+2x}{3},& \text{if } 0 < x \leq 1\\
    \frac{3-e^{-2x}}{3},& \text{if } 1 < x < \infty
\end{cases}
\end{equation}

The PDF is the following: 

\begin{equation}
f(x)= 
\begin{cases}
    \frac{2}{5}(e^{-2x}+2),& \text{if } 0 < x \leq 1\\
    \frac{2}{5}e^{-2x},& \text{if } 1 < x < \infty
\end{cases}
\end{equation}

First we need to generate a random variable that spans the positive real axis. We apply the exponential distribution as a known distribution $q(x) = e^{-x}$ and use acceptance-rejection method to generate the desired distribution using the PDF in Eq. (2). 
\begin{enumerate}
	\item Generate $q(y)$ by using $y = -\log{u}$ where $u \sim U[0,1]$. 
	\item Determine the constants $c_1$ and $c_2$ for first and second conditions. 
	\begin{equation}
		c_1 = \max_{0<t\leq 1} \frac{f(t)}{q(t)} = \frac{4e}{5(e-1)}\sqrt{2}
	\end{equation}
	\begin{equation}
		c_2 = \max_{1<t<\infty} \frac{f(t)}{q(t)} = \frac{2}{5}
	\end{equation}
	We can choose $c = \max(c_1, c_2)$ for both conditions. 
	\item Generate $u_2 \sim U[0,1]$ and set $x = y$ if the following holds
	\begin{equation}
		u_2 < \frac{f(y)}{cq(y)}
	\end{equation}
	otherwise go back to step one. 
\end{enumerate}

We generated $n = 1000000$ random samples and plotted the cumulative histogram for the samples as shown in Fig. 1. 

\begin{figure}
	\begin{center}
		\includegraphics[width=4in]{q17.png}
		\caption{Problem 17: The cumulative histogram of $n=100000$ samples and the real CDF as in Eq. (1).}
	\end{center}
\end{figure}





\section*{Problem 20}

We would like to generate random variables from CDF: 
\begin{equation}
	F(x) = \prod_{i=1}^n F_i(x)
\end{equation}
where $F_i(x)$'s are known and easy to generate. From Eq. (6), we have the following
\begin{equation}
\begin{split}
F(t) & = P(x \leq t) = \prod_{i=1}^n F_i(x) \\
 & = \prod_{i=1}^n P(x_i \leq t) = P(\max_i x_i \leq t)
\end{split}
\end{equation}
The CDF in Eq. (6) is the distribution of $\max_i x_i$ where $x_i \sim F_i(x_i)$. Therefore we can use the following steps to generate $F(x)$. 

\begin{enumerate}
	\item generate $u \sim U[0, 1]$
	\item use $u$ to generate $x_i$'s based on the corresponding CDFs $F_i$'s. 
	\item set $x = \max_i x_i$ 
\end{enumerate}



\section*{Problem 26}

The second-order transition probability $P(X_{k+1}|X_k, X_{k-1})$ where $P: {S \bigotimes  S^2} \to [0, 1]$. $S$ is the state space of the discrete Markov Chain. And we have 
\begin{equation}
\sum_{X_{k-1}\in S}\sum_{X_k\in S} P(X_{k+1}|X_k, X_{k-1}) = 1
\end{equation}

The transition probability $P(X_{k+1}|X_k, X_{k-1})$ can be re-written as

\begin{equation}
P(X_{k+1}|X_k, X_{k-1}) = P(X_{k+1}, X_k|X_k, X_{k-1})
\end{equation}
since it is conditioned on $X_k$. Now the transition probability can be viewed as a first order on the new expanded state space: $S^2$, i.e. $P: {S^2 \bigotimes  S^2} \to [0, 1]$. Moreover, we have 

\begin{equation}
\begin{split}
\sum_{\{X_k, X_{k-1}\}\in S^2} P(X_{k+1}, X_k|X_k, X_{k-1}) & = \sum_{X_{k-1}\in S}\sum_{X_k\in S} P(X_{k+1}, X_k|X_k, X_{k-1}) \\
 & = \sum_{X_{k-1}\in S}\sum_{X_k\in S} P(X_{k+1}|X_k, X_{k-1})\\
 & = 1
\end{split}
\end{equation}
Therefore, a second-order Markov Chain can be expressed as a first-order Markov Chain on the expanded state space: $S^2$. 


\section*{Problem 27}

\subsection*{(a)}

The implementation of inverse transform algorithm is the combination of the MATLAB function \textit{inverse\_transform$(a,b)$} and \textit{$itmc(n)$} where $P(state 1) = a$ and $P(state 1 || state 2) = b$ is the CDF. $n$ is the number of generated samples, i.e. $n=500$. The histogram of the generated samples is shown in the left plot of Fig. 2 with stationary distribution indicated by the red dots. 

\subsection*{(b)}

The implementation of the acceptance rejection algorithm is the combination of the MATLAB function \textit{acceptance\_rejection$(a,b)$} and \textit{$armc(n)$} using the same notation as in (a). The histogram of the generated samples is shown in the middle plot of Fig. 2 with stationary distribution indicated by the red dots. 


\subsection*{(c)}

With the empirical sequence of the Markov Chain: $(1222231331 \dots)$, we can estimate the transition matrix $p$. We used the MATLAB function \textit{strfind} to look for a specific pattern in the sequence. The transition probability from $i \to j$ is then the number of occurrences of pattern $(i, j)$ divided by number of occurrences of $(i)$, i.e.

\begin{equation}
p_{ij} = \frac{\#(i, j)}{\#(i)}
\end{equation}

The implementation is the MATLAB function \textit{estimatemc(x)} where $x$ is the sequence of the empirical samples.

The empirical transition matrix $p$ is found to be the following and different sampling might result in slightly different $p$. 

\begin{equation}
p = \begin{bmatrix}
0.1833 & 0.3833 & 0.4333 \\
0.1209 & 0.8791 & 0.0000 \\
0.0896 & 0.1045 & 0.8060
\end{bmatrix}
\end{equation}

\subsection*{(d)}

With the transition matrix $p$ we can obtain the empirical stationary distribution as 
\begin{equation}
\pi^*_\infty = \begin{bmatrix}
0.1200 & 0.6120 &0.2680
\end{bmatrix}',
\end{equation}

while the real stationary distribution is 
\begin{equation}
\pi_\infty = \begin{bmatrix}
0.1250 & 0.6250 &0.2500
\end{bmatrix}',
\end{equation}

Both $\pi^*_\infty$ and $\pi_\infty$ are shown in the right plot of Fig. 2 in magenta circles and red dots respectively. Good agreement between empirical and real stationary distributions can be seen. 


\begin{figure}
	\begin{center}
		\includegraphics[width=7in]{q27.png}
		\caption{Problem 27: 500 samples of the Marhov Chain using inverse transform (left) and acceptance-rejection (middle). The red dots represent the stationary distribution $\pi_\infty$ from the transition probability matrix $A$. The empirical estimator of a the transition matrix from the generated samples (right) where the magenta circles denotes the empirical stationary distribution $\pi^*_\infty$. }
	\end{center}
\end{figure}
\section*{Problem 28}

Without lost of generality, assume the 2-state transition probability is of the following form. 
\begin{equation}
P = \begin{bmatrix}
1-a & a \\
b & 1-b
\end{bmatrix}
\end{equation}
where $0 \leq a, b \leq 1$. With the stationary dist
ribution $\pi_\infty$ satisfying 
\begin{equation}
\pi_\infty = P' \pi_\infty = \frac{1}{a+b}\begin{bmatrix}
b \\
a
\end{bmatrix}
\end{equation}
Therefore, $\pi_\infty$ is the eigenvector of $P'$ corresponding to eigenvalue $1$. The other eigenvalue is $1-a-b$ corresponding to the eigenvector $[1, -1]'$. Given an arbitrary starting probability distribution 
\begin{equation}
\pi_0 = \begin{bmatrix}
c \\
1-c
\end{bmatrix} = \frac{1}{a+b}\begin{bmatrix}
b \\
a
\end{bmatrix} + \left(c - \frac{b}{a+b} \right)\begin{bmatrix}
1 \\
-1
\end{bmatrix} = \pi_\infty + \left(c - \frac{b}{a+b} \right)\begin{bmatrix}
1 \\
-1
\end{bmatrix}
\end{equation}
where $0 \leq c \leq 1$. Therefore, $\pi_k = (P')^k \pi_0$
\begin{equation}
\begin{split}
\pi_k & = (P')^k \pi_0 \\ 
 & = \pi_\infty + (1-a-b)^k\left(c - \frac{b}{a+b} \right)\begin{bmatrix}
1 \\
-1
\end{bmatrix}
\end{split}
\end{equation}
Therefore, $\pi_k$ converges to $\pi_\infty$ geometrically fast in terms of the second largest eigenvalue $1-a-b$. $\forall a,b$ satisfying $0 < a, b < 1$, $0 \leq |1-a-b| < 1$. There are a few situations to consider when $|1-a-b| = 1$. 
\begin{enumerate}
	\item $a=b=1$: the chain is periodic, $\pi_\infty$ does not exist unless $c=\frac{1}{2}$. 
	\item $a=b=0$: $P=I$ so $\pi_0 = \pi_\infty$.
\end{enumerate}

For a Markov Chain with three states, we can randomly generate the transition probability $P$ and do a numerical experiment. For a state space $S = \{1, 2, 3\}$ with state vector $s = [1, 2, 3]'$, we calculate the stationary average $\mu = s'\pi_\infty$. We initialize a random initial state $\pi_0$ and we have $\pi_n = (P')^n\pi_0$. We compute the difference between the average at time $n$ and stationary average: 

\begin{equation}
d_n = |s'(\pi_\infty - \pi_n)|
\end{equation}
And we demonstrate that the difference $d_n$ converges geometrically fast in terms of $|\lambda_2|$: $d_n \sim \mathcal{O}(|\lambda_2|^n)$ as shown in Fig. 3. 

\begin{figure}
	\begin{center}
		\includegraphics[width=4in]{q28.png}
		\caption{Problem 28: The convergence of $d_n$ is geometrically fast in terms of $|\lambda_2|$. }
	\end{center}
\end{figure}

\section*{Problem 30}

Assuming that $i \neq j$, we can remove state $j$ from the state space of the Markov Chain because we want to make sure that for time $k = 1, 2, 3 \dots n-1$, the chain does not visit state $j$. The new transition probability is now $R \in \mathbb{R}^{(n-1)\times (n-1)}$ with the $j$\textsuperscript{th} row and column removed. Moreover, we define the $P_{i\neq j, j} \in \mathbb{R}^{n-1}$ as the transition probability from state $i\neq j$ to state $j$. The starting state is $x_0 = i$, corresponding to a one-hot vector $\pi_0 \in \mathbb{R}^{n-1}$ with the $i$\textsuperscript{th} element equal to $1$. Let's list some of the quantities to see the underlying rules. 
\begin{equation}
P(x_1 = j | x_0 = i) = P_{ij} = P_{i\neq j, j}' \pi_0 = P_{i\neq j, j}' I \pi_0 
\end{equation}

\begin{equation}
P(x_2 = j, x_1 \neq j | x_0 = i) = P_{i\neq j, j}' R' \pi_0
\end{equation}

\begin{equation}
P(x_3 = j, x_2 \neq j, x_1 \neq j | x_0 = i) = P_{i\neq j, j}' (R')^2 \pi_0
\end{equation}

Therefore, we have 
\begin{equation}
P(x_n = j, x_{n-1} \neq j, \dots, x_1 \neq j | x_0 = i) = P_{i\neq j, j}' (R')^{n-1} \pi_0
\end{equation}
The probability of ending up in state $j$ given starting at state $i$ is then 
\begin{equation}
\begin{split}
p_{i\to j} & = \sum_{n=1}^{\infty} P(x_n = j, x_{n-1} \neq j, \dots, x_1 \neq j | x_0 = i)\\
 & = \sum_{n=1}^{\infty} P_{i\neq j, j}' (R')^{n-1} \pi_0 \\
 & = P_{i\neq j, j}' \left[ \sum_{n=1}^{\infty} (R')^{n-1} \right]\pi_0 \\
 & = P_{i\neq j, j}' (I - R')^{-1} \pi_0
\end{split}
\end{equation}
where $I$ is the identity matrix. The time it takes for the first visit of state $j$ given starting at state $i$, $T_{ij}$ is
\begin{equation}
\begin{split}
T_{ij} & = \sum_{n=1}^{\infty} n P(x_n = j, x_{n-1} \neq j, \dots, x_1 \neq j | x_0 = i)\\
 & = \sum_{n=1}^{\infty} P_{i\neq j, j}' n(R')^{n-1} \pi_0 \\
 & = P_{i\neq j, j}' \left[ \sum_{n=1}^{\infty} n(R')^{n-1} \right] \pi_0 \\
 & = P_{i\neq j, j}' \left[ (I - R')^{-1} \right]^2 \pi_0
\end{split}
\end{equation}


\section*{Problem 33}








\section*{Attachments}
	\textit{CDF.m, PDF.m, generate.m, q17.m, q28.m}

%\begin{thebibliography}{9}
%\bibitem{Robotics} Fred G. Martin \emph{Robotics Explorations: A Hands-On Introduction to Engineering}. New Jersey: Prentice Hall.
%\bibitem{Flueck}  Flueck, Alexander J. 2005. \emph{ECE 100}[online]. Chicago: Illinois Institute of Technology, Electrical and Computer Engineering Department, 2005 [cited 30
%August 2005]. Available from World Wide Web: (http://www.ece.iit.edu/~flueck/ece100).
%\end{thebibliography}

\end{document}
